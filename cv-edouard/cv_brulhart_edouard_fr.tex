%!TEX TS-program = xelatex
%!TEX encoding = UTF-8 Unicode
% Awesome CV LaTeX Template for CV/Resume
%
% This template has been downloaded from:
% https://github.com/posquit0/Awesome-CV
%
% Author:
% Claud D. Park <posquit0.bj@gmail.com>
% http://www.posquit0.com
%
% Template license:
% CC BY-SA 4.0 (https://creativecommons.org/licenses/by-sa/4.0/)
%


%-------------------------------------------------------------------------------
% CONFIGURATIONS
%-------------------------------------------------------------------------------
% A4 paper size by default, use 'letterpaper' for US letter
\documentclass[11pt, a4paper]{awesome-cv}

% Configure page margins with geometry
\geometry{left=1.4cm, top=.8cm, right=1.4cm, bottom=1.8cm, footskip=.5cm}

% Color for highlights
% Awesome Colors: awesome-emerald, awesome-skyblue, awesome-red, awesome-pink, awesome-orange
%                 awesome-nephritis, awesome-concrete, awesome-darknight
\colorlet{awesome}{awesome-red}
% Uncomment if you would like to specify your own color
% \definecolor{awesome}{HTML}{CA63A8}

% Colors for text
% Uncomment if you would like to specify your own color
% \definecolor{darktext}{HTML}{414141}
% \definecolor{text}{HTML}{333333}
% \definecolor{graytext}{HTML}{5D5D5D}
% \definecolor{lighttext}{HTML}{999999}
% \definecolor{sectiondivider}{HTML}{5D5D5D}

% Set false if you don't want to highlight section with awesome color
\setbool{acvSectionColorHighlight}{true}

% If you would like to change the social information separator from a pipe (|) to something else
\renewcommand{\acvHeaderSocialSep}{\quad\textbar\quad}

% If you would like to have locale-specific hyphenation and typographical rules
% for the language in which your CV is written, change the set language
\usepackage[french]{babel}

%-------------------------------------------------------------------------------
%	PERSONAL INFORMATION
%	Comment any of the lines below if they are not required
%-------------------------------------------------------------------------------
% Available options: circle|rectangle,edge/noedge,left/right
\photo{./profile.png}
\name{Edouard}{Brülhart}
\address{Route de la Scie-au-Crot 50, 1747 Corserey, Fribourg, Suisse}

\mobile{+41 79 290 71 19}
\email{edouard.brulhart@gmail.com}
\dateofbirth{26 mars 1998}
%\homepage{www.posquit0.com}
\github{edouardbruelhart}
\linkedin{edouardbrulhart}
% \gitlab{gitlab-id}
% \stackoverflow{SO-id}{SO-name}
% \twitter{@twit}
% \x{x-id}
% \skype{skype-id}
% \reddit{reddit-id}
% \medium{medium-id}
% \kaggle{kaggle-id}
% \hackerrank{hackerrank-id}
% \telegram{telegram-username}
% \googlescholar{googlescholar-id}{name-to-display}
%% \firstname and \lastname will be used
% \googlescholar{googlescholar-id}{}
% \extrainfo{extra information}

\quote{Efficace~~~·~~~Persévérant~~~·~~~Curieux~~~·~~~Déterminé~~~·~~~Consciencieux}
\position{Software Engineer - Interopérabilité}

%-------------------------------------------------------------------------------
\begin{document}

% Print the header with above personal information
% Give optional argument to change alignment(C: center, L: left, R: right)
\makecvheader

% Print the footer with 3 arguments(<left>, <center>, <right>)
% Leave any of these blank if they are not needed
\makecvfooter
  {\today}
  {Edouard Brülhart~~~·~~~Curriculum Vitae}
  {\thepage}


%-------------------------------------------------------------------------------
%	CV/RESUME CONTENT
%	Each section is imported separately, open each file in turn to modify content
%-------------------------------------------------------------------------------
%-------------------------------------------------------------------------------
%	SECTION TITLE
%-------------------------------------------------------------------------------
\cvsection{Compétences}


%-------------------------------------------------------------------------------
%	CONTENT
%-------------------------------------------------------------------------------
\begin{cvskills}

  \begin{minipage}[t]{0.45\textwidth}
    \cvskill
      {Développer des applications, services et pipelines pour la gestion et l'analyse de données scientifiques}
  \end{minipage}\hfill
  \begin{minipage}[t]{0.45\textwidth}
    \cvskill
      {Mettre en place un système de gestion de base de données relationnelle, SQL et NoSQL}
  \end{minipage}

  \begin{minipage}[t]{0.45\textwidth}
    \cvskill
      {Implémenter des interfaces de connection à des objets connectés (imprimantes, lecteurs RFID, etc.)}
  \end{minipage}\hfill
  \begin{minipage}[t]{0.45\textwidth}
    \cvskill
      {Administrer des serveurs Linux et déployer des services et applications en utilisant Docker}
  \end{minipage}

  \begin{minipage}[t]{0.45\textwidth}
    \cvskill
      {Gérer l'environnement de développement en utilisant Git, GitHub et Github Actions}
  \end{minipage}\hfill
  \begin{minipage}[t]{0.45\textwidth}
    \cvskill
      {Traiter et visualiser des données biologiques et environnementales}
  \end{minipage}

  \begin{minipage}[t]{0.45\textwidth}
    \cvskill
      {Créer une documentation précise et complète pour les utilisateurs et les développeurs}
  \end{minipage}\hfill
  \begin{minipage}[t]{0.45\textwidth}
    \cvskill
      {Travailler en équipe et collaborer efficacement avec des collègues de différents horizons}
  \end{minipage}

  \begin{minipage}[t]{0.45\textwidth}
    \cvskill
      {Former, assister et évaluer des étudiants et des collègues sur des sujets techniques}
  \end{minipage}\hfill
  \begin{minipage}[t]{0.45\textwidth}
    \cvskill
      {Collecter, traiter et analyser des échantillons biologiques et environnementaux}
  \end{minipage}

%---------------------------------------------------------
\end{cvskills}
%-------------------------------------------------------------------------------
%	SECTION TITLE
%-------------------------------------------------------------------------------
\cvsection{Erfahrung}


%-------------------------------------------------------------------------------
%	CONTENT
%-------------------------------------------------------------------------------
\begin{cventries}

%---------------------------------------------------------

\cventry
  {Nachwuchswissenschaftler – Befristeter Vertrag} % Job title
  {Universität Fribourg, Earth Metabolome Initiative} % Organization
  {Fribourg, Schweiz} % Location
  {2025} % Date(s)
  {}

%---------------------------------------------------------
\cventry
  {Nachwuchswissenschaftler – Befristeter Vertrag} % Job title
  {Universität Fribourg, Earth Metabolome Initiative} % Organization
  {Fribourg, Schweiz} % Location
  {2024} % Date(s)
  {}

  %---------------------------------------------------------
\cventry
  {Assistent der Metabolomik-Plattform – Befristeter Vertrag} % Job title
  {Universität Fribourg, MAPP Metabolomics} % Organization
  {Fribourg, Schweiz} % Location
  {2023} % Date(s)
  {}

%---------------------------------------------------------
\cventry
  {Masterarbeit – Manager in Samples and Collection} % Job title
  {Universität Fribourg, Earth Metabolome Initiative} % Organization
  {Fribourg, Schweiz} % Location
  {2022 -- 2024} % Date(s)
  {}

%---------------------------------------------------------
\cventry
  {Sicherheitsmitarbeiter – Teilzeit} % Job title
  {Phoenix Security} % Organization
  {Belfaux, Schweiz} % Location
  {2019 -- 2022} % Date(s)
  {}

%---------------------------------------------------------
\end{cventries}


\begin{minipage}[t]{0.48\textwidth}
  %-------------------------------------------------------------------------------
%	SECTION TITLE
%-------------------------------------------------------------------------------
\cvsection{Education}


%-------------------------------------------------------------------------------
%	CONTENT
%-------------------------------------------------------------------------------
\begin{cventries}

%---------------------------------------------------------
  \cventry
    {Master of Science in Environmental Biology, Option Plant and Microbial Sciences} % Degree
    {University of Fribourg} % Institution
    {Fribourg, Switzerland} % Location
    {2024} % Date(s)
    {
      \begin{cvitems} % Description(s) bullet points
        \item {Won the Syngenta Crop Protection price 2024 for the best average grade of the promotion -- 5.47/6}
      \end{cvitems}
    }

%---------------------------------------------------------
  \cventry
    {Bachelor of Science in Geography} % Degree
    {University of Fribourg} % Institution
    {Fribourg, Switzerland} % Location
    {2022} % Date(s)
    {}

%---------------------------------------------------------
\end{cventries}

\end{minipage}\hfill
\begin{minipage}[t]{0.48\textwidth}
  %-------------------------------------------------------------------------------
%	SECTION TITLE
%-------------------------------------------------------------------------------
\cvsection{Zusätzliche Aktivitäten}


%-------------------------------------------------------------------------------
%	CONTENT
%-------------------------------------------------------------------------------
\begin{cventries}

%---------------------------------------------------------
    \cventry
    {Full-Stack-Entwickler} % Affiliation/role
    {Manexp} % Organization/group
    {Schweiz} % Location
    {2024 - AKTUELL} % Date(s)
    {
      \begin{cvitems} % Description(s) of experience/contributions/knowledge
        \item {Entwicklung einer Progressive Web Application -- PWA zur Verwaltung einer Farm.}
        \item {Link: \href{https://manexp.ch}{https://manexp.ch}}
      \end{cvitems}
    }

%---------------------------------------------------------
    \cventry
    {Heimbrauer} % Affiliation/role
    {Brasserie de la Scie-au-Crot} % Organization/group
    {Schweiz} % Location
    {2020 - AKTUELL} % Date(s)
    {}

%---------------------------------------------------------
\end{cventries}

\end{minipage}

\begin{minipage}[t]{0.48\textwidth}
  %-------------------------------------------------------------------------------
%	SECTION TITLE
%-------------------------------------------------------------------------------
\cvsection{Softwares}


%-------------------------------------------------------------------------------
%	CONTENT
%-------------------------------------------------------------------------------
\begin{cvpairs}

%---------------------------------------------------------
  
\cvpair
    {General} % Category
    {MSOffice, LibreOffice} % Softwares

%---------------------------------------------------------
  
\cvpair
    {Programming} % Category
    {Visual Studio Code, IntelliJ Idea, Android Studio} % Softwares

%---------------------------------------------------------
  
\cvpair
    {Biology} % Category
    {Xcalibur, Proteowizard, MS Convert, Cytoscape} % Softwares

%---------------------------------------------------------
  
\cvpair
    {Geography} % Category
    {QGIS, ArcGIS, Structure From Motion} % Softwares

%---------------------------------------------------------
\end{cvpairs}
\end{minipage}\hfill
\begin{minipage}[t]{0.48\textwidth}
  %-------------------------------------------------------------------------------
%	SECTION TITLE
%-------------------------------------------------------------------------------
\cvsection{Sprachen}


%-------------------------------------------------------------------------------
%	CONTENT
%-------------------------------------------------------------------------------
\begin{cvpairs}

%---------------------------------------------------------
  
\cvpair
    {Französisch} % Category
    {Muttersprache} % Level

%---------------------------------------------------------

\cvpair
    {Englisch} % Category
    {C1, Täglich} % Level

%---------------------------------------------------------

\cvpair
    {Deutsch} % Category
    {B2, Regelmäßig} % Level

%---------------------------------------------------------
\end{cvpairs}
\end{minipage}

\begin{minipage}[t]{0.48\textwidth}
  %-------------------------------------------------------------------------------
%	SECTION TITLE
%-------------------------------------------------------------------------------
\cvsection{Programming}


%-------------------------------------------------------------------------------
%	CONTENT
%-------------------------------------------------------------------------------
\begin{cvpairs}

%---------------------------------------------------------
  
\cvpair
    {Languages} % Category
    {Kotlin, LaTeX, Python, R, Rust, SQL} % Skill

%---------------------------------------------------------

\cvpair
    {DevOps} % Category
    {Docker, Git, Bash, Google Cloud} % Skill

%---------------------------------------------------------
\end{cvpairs}
\end{minipage}\hfill
\begin{minipage}[t]{0.48\textwidth}
  %-------------------------------------------------------------------------------
%	SECTION TITLE
%-------------------------------------------------------------------------------
\cvsection{Interessen}


%-------------------------------------------------------------------------------
%	CONTENT
%-------------------------------------------------------------------------------
\begin{cvpairs}

%---------------------------------------------------------
  
\cvpair
    {Sport} % Category
    {Fussball, Mountainbike, Ski} % Interests

%---------------------------------------------------------

\cvpair
    {Biologie} % Category
    {Entomologie, Botanik} % Interests

%---------------------------------------------------------
\end{cvpairs}
\end{minipage}

%-------------------------------------------------------------------------------
\end{document}
